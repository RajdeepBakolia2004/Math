\documentclass{article}

\usepackage[utf8]{inputenc}
\usepackage{amsmath, amsthm, amsfonts}
\usepackage{graphicx}
\usepackage{float}
\title{lecture 2}
\author{Rajdeep Bakolia}
\date{\today}

\newtheorem{theorem}{Theorem}[section]
% you can add optional [] to give whfere numbering should happen
\newtheorem{corollary}{Corollary}[theorem]


\newcommand{\R}{\mathbb{R}}
\newcommand{\N}{\mathbb{N}}
\newcommand{\Q}{\mathbb{Q}}

\newcommand{\bv}[6]{\begin{pmatrix}
    #1 & #2 & #3\\
    #4 & #5 & #6\\
\end{pmatrix}}

\begin{document}
\maketitle

\section*{definition of $e$}

\begin{enumerate}
\item As a \textbf{Limit}:
\begin{equation} \label{limit}
\begin{split}
e &= \lim_{n \to \infty} \left ( 1 + \frac{1}{n}\right )^n\\
&= \lim_{t \to 0} \left ( 1 + t \right )^{\frac{1}{t}}\\
\end{split}
\end{equation}
\item As a \textit{Sum}:
\begin{align}
    \label{sum1}
    e =  \sum_{n = 0}^{\infty} \frac{1}{n!}\\
    \label{sum2}
    e = \sum_{t = 1}^{\infty} \frac{1}{(t-1)!}
\end{align}

\item As a \underline{continued fraction}
\[e = 2 + \frac{1}{1 + \frac{1}{2 + \frac{2}{3 + \frac{3}{4 + \frac{4}{5 + \ddots}}}}}\]
\end{enumerate}
Equation \ref{limit} was really cool.\\
Equation \ref{sum1} is same as \ref{sum2}\\
\begin{multline}
    e^{x} = 1 + \frac{x}{1!} + \frac{x^2}{2!} + \frac{x^3}{3!} + \frac{x^4}{4!}+ \frac{x^5}{5!} + \frac{x^6}{6!} +\frac{x^7}{7!} + \frac{x^8}{8!}\\ + \frac{x^9}{9!} + \frac{x^{10}}{10!} + \frac{x^{11}}{11!} + \frac{x^{12}}{12!} + \frac{x^{13}}{13!} + \frac{x^{14}}{14!} + \frac{x^{15}}{15!} + \frac{x^{16}}{16!} 
\end{multline}


\newpage


\section*{More tricks}


\begin{table}[H]
\caption{A simple Table}
\label{t1}
\begin{center}
\begin{tabular}{|c | l | r | c |}
    \hline
    1 & 2 & 8 & 9 \\
    \hline
    3 & 5000000000 & 90000000 & 0\\
    \hline
    
\end{tabular}
\end{center}
\end{table}
I like table \ref{t1}

\begin{figure}[H]
    \centering
    \includegraphics[width = \textwidth]{w }
    \caption{Image display}
    \label{img1}
\end{figure}
figure \ref{img1} is a pokemon named pikachu.\\
tabular and includegraphics are actual code for both \\
table and images the table and figure is used to give caption and label to them\\
It creates a float object and latex try to find best place to fit that\\
to resolve this us the package float and add [H] as shown.


\newpage
\section{Theorem and Macros}

\begin{theorem}
    sum of first n natural number is $\frac{n(n+1)}{2}$
\end{theorem}
\begin{proof}
    \begin*{itemize}
        \item Note that,
        \item $\sum_{i = 1}^{n}i \text{ and } \sum_{i = 1}^{n}(n-i+1)$
        \item Both are equal adding these gives
        \item $n(n+1)$
        \item and then divide the equation by two to get the sum 
    \end*{itemize}
    
    
    
    
    
\end{proof}

\begin{corollary}
    let $m, n \in \N, m < n$ then sum from m + 1 to n is $\frac{n(n+1) - m(m+1)}{2}$
\end{corollary}

\begin{theorem}
    in standard notation  $\N \subset \Q \subset \R$
\end{theorem}

\begin{corollary}
    $\N \cap \Q = \Q$ 
\end{corollary}
\begin{proof}
    as $\N$ is a subset of $\Q$
\end{proof}

$\bv{34}{56}{34}{76}{54}{45}$
\end{document}