\documentclass{article}
\usepackage{amsmath}
\usepackage{amsfonts}
\usepackage{amssymb}
\begin{document}
\title{Linear Space}
\author{Rajdeep Bakolia}
\date{\today}
\maketitle

\section{Introduction}
In defining linear space we define the axims the elements of the space must satisfy.
We do not define the elements or the operations on them.
example: \( \mathbb{R}^n, \mathbb{R}, \mathbb{C} \), polynomials, etc.
\section{Definition}
let $V$ be a non-empty set of objects called elements. Then $V$ is
called a linear space if the following axioms are satisfied:
\begin{enumerate}
    \item $V$ is closed under addition. For all $x,y \in V$, there exist an element $x+y \in V$ called the sum of x and y.
    \item $V$ is closed under scalar multiplication. For all $x \in V$ and all scalars $c \in  \mathbb{R} $, there exists an element $cx \in V$ called the scalar product of c and x.
    \item Addition is commutative. For all $x,y \in V$, $x+y = y+x$.
    \item Addition is associative. For all $x,y,z \in V$, $(x+y)+z = x+(y+z)$.
    \item $V$ there exists an element zero $\mathcal{O}$ such that for all $x \in V$, $x+\mathcal{O} = x$.
    \item For each $x \in V$, there exists an element $-x$ such that $x+(-x) = \mathcal{O}$.
    \item Scalar multiplication is associative. For all $x \in V$ and all scalars $a,b \in \mathbb{R}$, $(ab)x = a(bx)$.
    \item Scalar multiplication distributes over vector addition. For all $x,y \in V$ and all scalars $a \in \mathbb{R}$, $a(x+y) = ax + ay$.
    \item Scalar multiplication distributes over scalar addition. For all $x \in V$ and all scalars $a,b \in \mathbb{R}$, $(a+b)x = ax + bx$.
    \item 1 is the identity of multiplication. For all $x \in V$, $1x = x$.
\end{enumerate}

The vector space or linear space which satify the above for all real number are called 
real vector space or real linear space.
The vector space or linear space which satify the above for all complex number are called
complex vector space or complex linear space.

\section{Theorem}
\textbf{Theorem 1:} The zero vector is unique. \newline

\textbf{Proof:} Let $\mathcal{O}_1$ and $\mathcal{O}_2$ be two zero vectors in $V$.

Then $\mathcal{O}_1 + \mathcal{O}_2 = \mathcal{O}_1$ and

$\mathcal{O}_1 + \mathcal{O}_2 = \mathcal{O}_2$.

Thus $\mathcal{O}_1 = \mathcal{O}_2$. \newline


\textbf{Theorem 2:} Uniqueness of additive inverse. \newline

\textbf{Proof:} Let $x$ be an element of $V$ and let $y_1$ and $y_2$ be two additive inverses of $x$.

Then $x + y_1 = \mathcal{O}$ and $x + y_2 = \mathcal{O}$.

$y_1 = y_1 + \mathcal{O} = y_1 + (x + y_2) = (y_1 + x) + y_2 = \mathcal{O} + y_2 = y_2$.

hence $y_1 = y_2$. \newline

note that -x = $(-1)x$ is the additive inverse of x. follows from uniqueness of distributive law.


\section{Properties of Linear Space}

\textbf{Theorem 3:}
Let $x, y \in V$ and $a, b \in \mathbb{R}$. Then:
\begin{enumerate}
    \item $0 \cdot x = \mathcal{O}$
    \item $a \cdot \mathcal{O} = \mathcal{O}$
    \item $(-a) \cdot x = -(a \cdot x) = a \cdot (-x)$
    \item If $a \cdot x = \mathcal{O}$, then either $a = 0$ or $x = \mathcal{O}$
    \item If $a \cdot x = a \cdot y$ and $a \neq 0$, then $x = y$
    \item If $a \cdot x = b \cdot x$ and $x \neq \mathcal{O}$, then $a = b$
    \item $-(x + y) = (-x) + (-y)$
    \item $x + x = 2x$, $x + x + x = 3x$, in general, $nx = x + x + \dots + x$ ($n$ times)
\end{enumerate}

\section*{Proof}

\subsection*{(1) $0 \cdot x = \mathcal{O}$}
\[
2(0 \cdot x) = (2 \cdot 0) \cdot x = 0 \cdot x
\]
Now, add $-1(0 \cdot x)$ to both sides, which is the inverse of $0 \cdot x$:
\[
0 \cdot x = \mathcal{O}
\]

\subsection*{(2) $a \cdot \mathcal{O} = \mathcal{O}$}
\[
a \cdot \mathcal{O} + a \cdot \mathcal{O} = a(\mathcal{O} + \mathcal{O}) = a \cdot \mathcal{O}
\]
Now, add $-1(a \cdot \mathcal{O})$ to both sides:
\[
0 \cdot x = \mathcal{O}
\]

\subsection*{(3) $(-a) \cdot x = -(a \cdot x) = a \cdot (-x)$}
\[
(-a) \cdot x + a \cdot x = (-a + a) \cdot x = 0 \cdot x = \mathcal{O}
\]
By uniqueness of inverses:
\[
(-a) \cdot x = (-1)(a \cdot x) = a(-x)
\]

\subsection*{(4) If $a \cdot x = \mathcal{O}$, then either $a = 0$ or $x = \mathcal{O}$}
Assume $a \neq 0$. Multiply both sides by $\frac{1}{a}$:
\[
\frac{1}{a} \cdot (a \cdot x) = \frac{1}{a} \cdot \mathcal{O}
\]
\[
(\frac{1}{a}a) \cdot x = \mathcal{O} \implies 1 \cdot x = \mathcal{O} \implies x = \mathcal{O}
\]

\subsection*{(5) If $a \cdot x = a \cdot y$ and $a \neq 0$, then $x = y$}
\[
a \cdot x - a \cdot y = \mathcal{O}
\]
\[
a(x - y) = \mathcal{O}
\]
Since $a \neq 0$, by (4), $x - y = \mathcal{O} \implies x = y$.

\subsection*{(6) If $a \cdot x = b \cdot x$ and $x \neq \mathcal{O}$, then $a = b$}
\[
a \cdot x - b \cdot x = \mathcal{O}
\]
\[
(a - b) \cdot x = \mathcal{O}
\]
Since $x \neq \mathcal{O}$, $a - b = 0 \implies a = b$.

\subsection*{(7) $-(x + y) = (-x) + (-y)$}
\[
(-1)(x + y) = (-1)x + (-1)y
\]
Add $(-x) + (-y)$ to $(x + y)$:
\[
(x + y) + (-x) + (-y) = (x + (-x)) + (y + (-y)) = \mathcal{O}
\]

\subsection*{(8) $nx = x + x + \dots + x$ ($n$ times)}
Proof by Induction \\
Base case For $n = 1$, $1x = x$. \newline
Inductive step Assume true for $n = k$, i.e., $kx = x + x + \dots + x$ ($k$ times). \\
For $n = k + 1$:
\[
(k + 1)x = kx + x = (x + x + \dots + x) + x \quad (k + 1 \text{ times})
\]
Thus, the statement holds for all $n \in \mathbb{N}$.

\end{document}
